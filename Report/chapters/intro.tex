\section{Overview}
	Emotion recognition from speech is important with tasks involving Human-Machine interaction in the same way it is important to tasks involving Human-Human interaction. Communication through speech is the quickest method of communication between humans. If this speed can be achieved by AI systems, it could revolutionize human-computer interaction. However, more research is still needed to reach the revolutionary point. The current main issues in this field are related to the search for highly discriminate features and appropriate classification methods to which this project aims to contribute to.
\section{Objective}
	The primary objective of this project was to build a system to automatically recognize emotions expressed in children's speech using acoustic speech data using the latest state of the art techniques of I Vectors and neural networks. In order to achieve this goal, the following steps were taken:
	\begin{enumerate}
		\item Conduct a review of current techniques and how they came to be used to understand potential optimizations or changes that can be made to improve recognition performance.
		\item Gain a clear understanding of the dataset used to better understand how results can be interpreted.
		\item Use and optimize detailed and efficient feature extraction and classification techniques while explaining the processes used in order to replicate and improve successful techniques.
		\item Evaluate the results and contribute to the field of emotion recognition with any interesting, new or contradictory findings, furthering the field of research.
	\end{enumerate}
	