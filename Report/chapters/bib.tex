\setlength{\parindent}{0cm}
\small
\hangindent=0.5cm
 Alex, S.B., Babu, B.P., Mary, L., 2018. Utterance and Syllable Level Prosodic Features for Automatic Emotion Recognition, in: 2018 IEEE Recent Advances in Intelligent Computational Systems (RAICS). Presented at the 2018 IEEE Recent Advances in Intelligent Computational Systems (RAICS), pp. 31–35. https://doi.org/10.1109/RAICS.2018.8635059

\small
\hangindent=0.5cm
 Alim, S.A., Rashid, N.K.A., 2018. Some Commonly Used Speech Feature Extraction Algorithms. From Natural to Artificial Intelligence - Algorithms and Applications. https://doi.org/10.5772/intechopen.80419

\small
\hangindent=0.5cm
 Al-Kaltakchi, M.T.S., Woo, W.L., Dlay, S.S., Chambers, J.A., 2017. Comparison of I-vector and GMM-UBM approaches to speaker identification with TIMIT and NIST 2008 databases in challenging environments, in: 2017 25th European Signal Processing Conference (EUSIPCO). Presented at the 2017 25th European Signal Processing Conference (EUSIPCO), pp. 533–537. \\https://doi.org/10.23919/EUSIPCO.2017.8081264

\small
\hangindent=0.5cm
Batliner, A., Blomberg, M., D’Arcy, S., Elenius, D., Giuliani, D., Gerosa, M., Hacker, C., Russell, M., Steidl, S., Wong, M., 2005. The PF\_STAR children’s speech corpus. pp. 2761–2764.

\small
\hangindent=0.5cm
 Batliner, A., Hacker, C., Steidl, S., Nöth, E., D’Arcy, S., Russell, M., Wong, M., 2004. “You Stupid Tin Box” - Children Interacting with the AIBO Robot: A Cross-linguistic Emotional Speech Corpus, in: Proceedings of the Fourth International Conference on Language Resources and Evaluation (LREC’04). Presented at the LREC 2004, European Language Resources Association (ELRA), Lisbon, Portugal.

\small
\hangindent=0.5cm
Batliner, A., Schuller, B., Seppi, D., Steidl, S., Devillers, L., Vidrascu, L., Vogt, T., Aharonson, V., Amir, N., 2011. The Automatic Recognition of Emotions in Speech, in: Cognitive Technologies. pp. 71–99. https://doi.org/10.1007/978-3-642-15184-2\_6

\small
\hangindent=0.5cm
Cabanac, M., 2002. What is emotion? Behavioural Processes 60, 69–83. \\https://doi.org/10.1016/S0376-6357(02)00078-5

\small
\hangindent=0.5cm
Casale, S., Russo, A., Scebba, G., Serrano, S., 2008. Speech Emotion Classification Using Machine Learning Algorithms, in: 2008 IEEE International Conference on Semantic Computing. Presented at the 2008 IEEE International Conference on Semantic Computing, pp. 158–165. \\https://doi.org/10.1109/ICSC.2008.43

\small
\hangindent=0.5cm
Chen, C.-Y., Chen, C.-P., 2016. Support Super-Vector Machines in Automatic Speech Emotion Recognition 11.

\small
\hangindent=0.5cm
Cowen, A.S., Keltner, D., 2017. Self-report captures 27 distinct categories of emotion bridged by continuous gradients. PNAS. https://doi.org/10.1073/pnas.1702247114

\small
\hangindent=0.5cm
Frick, R.W., 1985. Communicating emotion: The role of prosodic features. Psychological Bulletin 97, 412–429. https://doi.org/10.1037/0033-2909.97.3.412

\small
\hangindent=0.5cm
Gao, M., Dong, J., Zhou, D., Zhang, Q., Yang, D., 2019. End-to-End Speech Emotion Recognition Based on One-Dimensional Convolutional Neural Network, in: Proceedings of the 2019 3rd International Conference on Innovation in Artificial Intelligence, ICIAI 2019. Association for Computing Machinery, Suzhou, China, pp. 78–82. https://doi.org/10.1145/3319921.3319963

\small
\hangindent=0.5cm
Gupta, S., 2016. Application of MFCC in Text Independent Speaker Recognition.

\small
\hangindent=0.5cm
Han, K., Yu, D., Tashev, I., 2014. Speech Emotion Recognition Using Deep Neural Network and Extreme Learning Machine 5.

\small
\hangindent=0.5cm
Heracleous, P., Yoneyama, A., 2019. A comprehensive study on bilingual and multilingual speech emotion recognition using a two-pass classification scheme. PLoS ONE 14, e0220386. \\https://doi.org/10.1371/journal.pone.0220386

\small
\hangindent=0.5cm
Hu, H., Xu, M.-X., Wu, W., 2007. GMM Supervector Based SVM with Spectral Features for Speech Emotion Recognition, in: 2007 IEEE International Conference on Acoustics, Speech and Signal Processing - ICASSP ’07. Presented at the 2007 IEEE International Conference on Acoustics, Speech and Signal Processing - ICASSP ’07, pp. IV-413-IV–416. https://doi.org/10.1109/ICASSP.2007.366937

\small
\hangindent=0.5cm
Ibrahim, N.S., Ramli, D.A., 2018. I-vector Extraction for Speaker Recognition Based on Dimensionality Reduction. Procedia Computer Science, Knowledge-Based and Intelligent Information \& Engineering Systems: Proceedings of the 22nd International Conference, KES-2018, Belgrade, Serbia 126, 1534–1540. \\https://doi.org/10.1016/j.procs.2018.08.126

\small
\hangindent=0.5cm
Issa, D., Fatih Demirci, M., Yazici, A., 2020. Speech emotion recognition with deep convolutional neural networks. Biomedical Signal Processing and Control 59, 101894. \\https://doi.org/10.1016/j.bspc.2020.101894

\small
\hangindent=0.5cm
Jin, Q., Li, C., Chen, S., Wu, H., 2015. Speech emotion recognition with acoustic and lexical features, in: 2015 IEEE International Conference on Acoustics, Speech and Signal Processing (ICASSP). Presented at the 2015 IEEE International Conference on Acoustics, Speech and Signal Processing (ICASSP), pp. 4749–4753. https://doi.org/10.1109/ICASSP.2015.7178872

\small
\hangindent=0.5cm
Karafiát, M., Burget, L., Matejka, P., Glembek, O., Cernocky, J., 2011. iVector-based discriminative adaptation for automatic speech recognition. https://doi.org/10.1109/ASRU.2011.6163922

\small
\hangindent=0.5cm
Kinnunen, T., Li, H., 2010. An Overview of Text-Independent Speaker Recognition: from Features to Supervectors. Speech Communication 52, 12–40. https://doi.org/10.1016/j.specom.2009.08.009

\small
\hangindent=0.5cm
Kołakowska, A., Landowska, A., Szwoch, M., Szwoch, W., Wróbel, M., 2014. Emotion Recognition and Its Applications. Advances in Intelligent Systems and Computing 300, 51–62. https://doi.org/10.1007/978-3-319-08491-6\_5

\small
\hangindent=0.5cm
Lim, W., Jang, D., Lee, T., 2016. Speech emotion recognition using convolutional and Recurrent Neural Networks, in: 2016 Asia-Pacific Signal and Information Processing Association Annual Summit and Conference (APSIPA). Presented at the 2016 Asia-Pacific Signal and Information Processing Association Annual Summit and Conference (APSIPA), pp. 1–4. \\https://doi.org/10.1109/APSIPA.2016.7820699

\small
\hangindent=0.5cm
Markel, J., Oshika, B., Gray, A., 1977. Long-term feature averaging for speaker recognition. IEEE Transactions on Acoustics, Speech, and Signal Processing 25, 330–337.\\ https://doi.org/10.1109/TASSP.1977.1162961

\small
\hangindent=0.5cm
Mehta, D., Siddiqui, M.F.H., Javaid, A.Y., 2019. Recognition of Emotion Intensities Using Machine Learning Algorithms: A Comparative Study. Sensors (Basel) 19. https://doi.org/10.3390/s19081897

\small
\hangindent=0.5cm
Mesquita, B., Walker, R., 2003. Cultural differences in emotions: a context for interpreting emotional experiences. Behaviour Research and Therapy 41, 777–793. https://doi.org/10.1016/S0005-7967(02)00189-4

\small
\hangindent=0.5cm
Pérez-Espinosa, H., Martínez-Miranda, J., Espinosa-Curiel, I., Rodríguez-Jacobo, J., Villaseñor-Pineda, L., Avila-George, H., 2020. IESC-Child: An Interactive Emotional Children’s Speech Corpus. Computer Speech \& Language 59, 55–74. https://doi.org/10.1016/j.csl.2019.06.006

\small
\hangindent=0.5cm
Plutchik, R., 1982. A psychoevolutionary theory of emotions: Social Science Information. \\https://doi.org/10.1177/053901882021004003

\small
\hangindent=0.5cm
Rajan, R., G., H.U., C., S.A., M., R.T., 2019. Design and Development of a Multi-Lingual Speech Corpora (TaMaR-EmoDB) for Emotion Analysis, in: Interspeech 2019. Presented at the Interspeech 2019, ISCA, pp. 3267–3271. https://doi.org/10.21437/Interspeech.2019-2034

\small
\hangindent=0.5cm
Reddy, A.P., Vijayarajan, V., 2020. Audio compression with multi-algorithm fusion and its impact in speech emotion recognition. Int J Speech Technol. https://doi.org/10.1007/s10772-020-09689-9

\small
\hangindent=0.5cm
Rizzo, A., Morie, J.F., Williams, J., Pair, J., Buckwalter, J.G., 2005. Human Emotional State and its Relevance for Military VR Training. https://doi.org/null

\small
\hangindent=0.5cm
Russell, M., 2006. The PF-STAR British English Children’s Speech Corpus 35.

\small
\hangindent=0.5cm
Sabini, J., Silver, M., 2005. Ekman’s basic emotions: Why not love and jealousy? Cognition \& Emotion 19, 693–712. https://doi.org/10.1080/02699930441000481

\small
\hangindent=0.5cm
Sadjadi, S.O., Slaney, M., Heck, L., 2013. MSR Identity Toolbox.

\small
\hangindent=0.5cm
Schwenker, F., Scherer, S., Magdi, Y.M., Palm, G., 2009. The GMM-SVM Supervector Approach for the Recognition of the Emotional Status from Speech, in: Alippi, C., Polycarpou, M., Panayiotou, C., Ellinas, G. (Eds.), Artificial Neural Networks – ICANN 2009, Lecture Notes in Computer Science. Springer, Berlin, Heidelberg, pp. 894–903. https://doi.org/10.1007/978-3-642-04274-4\_92

\small
\hangindent=0.5cm
Wierzbicka, A., 1992. Defining emotion concepts. Cognitive Science 16, 539–581. \\https://doi.org/10.1016/0364-0213(92)90031-O

\small
\hangindent=0.5cm
Xia, R., Liu, Y., 2016. DBN-ivector Framework for Acoustic Emotion Recognition. Presented at the Interspeech 2016, pp. 480–484. https://doi.org/10.21437/Interspeech.2016-488

